%%%%%%%%%%%%%%%%%%%%%%%%%%%%%%%%%%%%%%%%%%
% Dokument
%%%%%%%%%%%%%%%%%%%%%%%%%%%%%%%%%%%%%%%%%%
% Geometrie
\newcommand{\paperFormat}{a4paper}
\newcommand{\lPageMargin}{25mm}
\newcommand{\rPageMargin}{20mm}
\newcommand{\tPageMargin}{20mm}
\newcommand{\bPageMargin}{20mm}

\documentclass[11pt,oneside]{scrartcl}

\newcommand{\newpar}{\par\par}

\usepackage[pdftitle={\titleinfo},%
			pdfauthor={\authorinfo},%
			pdfcreator={pdfLatex, LaTeX with hyperref},
			pdfsubject={\subjectinfo},
			plainpages=false,
			pdfpagelabels,
			colorlinks,
			linkcolor=black,
			filecolor=black,
			citecolor=black,
			urlcolor=black]{hyperref}
			
\usepackage[scaled]{helvet}
%\renewcommand\familydefault{\sfdefault} 


% Headings
\usepackage{scrlayer-scrpage}

%%%%%%%%%%%%%%%%%%%%%%%%%%%%%%%%%%%%%%%%%%
% Package's
%%%%%%%%%%%%%%%%%%%%%%%%%%%%%%%%%%%%%%%%%%
\usepackage[acronym]{glossaries}

\usepackage[OT1]{fontenc}

\usepackage[utf8]{inputenc}

\usepackage[free-standing-units=true,use-xspace=true]{siunitx}

\usepackage{layout}
\setlength{\parindent}{0em}

\renewcommand{\baselinestretch}{1.2}
\renewcommand{\arraystretch}{1}
\newcolumntype{P}[1]{>{\centering\arraybackslash}p{#1}}

%% This changes default fonts for both text and math mode to use Herman Zapfs
%% excellent Palatino font.  Do not change this.
%\usepackage[sc]{mathpazo}

%% The AMS-LaTeX extensions for mathematical typesetting.  Do not
%% remove.
\usepackage{amsmath,amssymb,amsfonts,mathrsfs}

\usepackage[capitalize, noabbrev]{cleveref}	%cref starts with capital letter
\usepackage[usenames,dvipsnames]{pstricks}
\usepackage{setspace}
\usepackage{epsfig}
\usepackage{pst-pdf}
\usepackage{pst-all}
\usepackage{pstricks-add}
\usepackage{supertabular}
\usepackage[font=small,labelfont=bf]{caption}
\usepackage[font=footnotesize]{subfig}
\usepackage{footnote}
\usepackage{float}
\usepackage{multirow}
\usepackage{pdfpages}
\usepackage{pgf,tikz}
\usepackage{color}
\usepackage{titletoc}

\usepackage[makeroom]{cancel}
\usepackage{array}
\usepackage{trfsigns}
\usepackage{textcomp}
\usepackage{booktabs}
\usepackage{rotating}

\usepackage{listings}

\usepackage{tabto}
\renewcommand{\captionfont}{\scriptsize\slshape}

%Inhaltsverzeichnis
\setcounter{secnumdepth}{4}
\setcounter{tocdepth}{2}

%Geometrie
\usepackage[\paperFormat,left=\lPageMargin,right=\rPageMargin,top=\tPageMargin,bottom=\bPageMargin,includeheadfoot]{geometry}

\usepackage{lipsum}%dummy text only
\usepackage{tikz}
\usetikzlibrary{fadings}
\newcommand{\gradient}{\noindent%
    \begin{tikzpicture}
    \fill[black,path fading=west] (-0.5\linewidth,0) rectangle (0,0.1ex);
    \fill[black,path fading=east] (0,0) rectangle (0.5\linewidth,0.1ex);
    \end{tikzpicture}%
}

%%%%%%%%%%%%%%%%%%%%%%%%%%%%%%%%%%%%%%%%%%%%%%%%%%%%%%%%%%%%%%%%
% Environment Numbering
%%%%%%%%%%%%%%%%%%%%%%%%%%%%%%%%%%%%%%%%%%%%%%%%%%%%%%%%%%%%%%%%

%Abbildungsnumerierung anhand Kapitel
\renewcommand{\thefigure}{\arabic{section}.\arabic{figure}}
\makeatletter \@addtoreset{figure}{section} \makeatother

%Gleichungen anhand Kapitel
\AtBeginDocument{\numberwithin{equation}{section}}
\AtBeginDocument{\numberwithin{figure}{section}}
\AtBeginDocument{\numberwithin{table}{section}}


%%%%%%%%%%%%%%%%%%%%%%%%%%%%%%%%%%%%%%%%%%%%%%%%%%%%%%%%%%%%%%%%
% Farben
%%%%%%%%%%%%%%%%%%%%%%%%%%%%%%%%%%%%%%%%%%%%%%%%%%%%%%%%%%%%%%%%
\definecolor{black}{rgb}{0,0,0}
\definecolor{red}{rgb}{1,0,0}
\definecolor{white}{rgb}{1,1,1}
\definecolor{grey}{rgb}{0.8,0.8,0.8}
\definecolor{bgGray}{rgb}{0.95,0.95,0.95}
\definecolor{stringColor}{rgb}{0.16,0.00,1.00}
\definecolor{annotationColor}{rgb}{0.39,0.39,0.39}
\definecolor{keywordColor}{rgb}{0.50,0.00,0.33}
\definecolor{commentColor}{rgb}{0.25,0.50,0.37}

%%%%%%%%%%%%%%%%%%%%%%%%%%%%%%%%%%%%%%%%%%%%%%%%%%%%%%%%%%%%%%%%
% Listing Styles
%%%%%%%%%%%%%%%%%%%%%%%%%%%%%%%%%%%%%%%%%%%%%%%%%%%%%%%%%%%%%%%%
\lstdefinestyle{bash}{
  language=bash,
  basicstyle=\normalsize\ttfamily,
  backgroundcolor = \color{bgGray},
  xleftmargin = 0cm,
  xrightmargin = 0cm,
  framexleftmargin = 0em,
  frame=tb,
  showstringspaces=false
}

\lstdefinestyle{cpp}{
		%linebackgroundcolor={\ifodd\value{lstnumber}\color{bgGray}\else\color{white}\fi},   % choose the background color; you must add \usepackage{color} or \usepackage{xcolor}; should come as last argument
	backgroundcolor=\color{bgGray},
	basicstyle=\normalsize\ttfamily,        % the size of the fonts that are used for the code
	breakatwhitespace=false,         % sets if automatic breaks should only happen at whitespace
	breaklines=true,                 % sets automatic line breaking
	captionpos=b,                    % sets the caption-position to bottom
	commentstyle=\color{commentColor},    % comment style
	deletekeywords={...},            % if you want to delete keywords from the given language
	escapeinside={\%*}{*)},          % if you want to add LaTeX within your code
	extendedchars=true,              % lets you use non-ASCII characters; for 8-bits encodings only, does not work with UTF-8
	frame=tb,	                  	 % adds a frame around the code
	keepspaces=true,                 % keeps spaces in text, useful for keeping indentation of code (possibly needs columns=flexible)
	keywordstyle=\color{keywordColor}\bfseries,   % keyword style
	language=C++,                    % the language of the code
	morekeywords={*,...},            % if you want to add more keywords to the set
	numbers=none,                    % where to put the line-numbers; possible values are (none, left, right)
	numbersep=3pt,                   % how far the line-numbers are from the code
	numberstyle=\footnotesize\color{codeGray}, % the style that is used for the line-numbers
	rulecolor=\color{black},         % if not set, the frame-color may be changed on line-breaks within not-black text (e.g. comments (green here))
	showspaces=false,                % show spaces everywhere adding particular underscores; it overrides 'showstringspaces'
	showstringspaces=false,          % underline spaces within strings only
	showtabs=false,                  % show tabs within strings adding particular underscores
	stringstyle=\color{stringColor},     % string literal style
	tabsize=2,	                   % sets default tabsize to 2 spaces
	title=\lstname                   % show the filename of files included with \lstinputlisting; also try caption instead of title}
}


%%%%%%%%%%%%%%%%%%%%%%%%%%%%%%%%%%%%%%%%%%%%%%%%%%%%%%%%%%%%%%%%
% Einheiten
%%%%%%%%%%%%%%%%%%%%%%%%%%%%%%%%%%%%%%%%%%%%%%%%%%%%%%%%%%%%%%%%
%\usepackage[Gray,squaren]{SIunits} %\gray befehl heisst nun \Gray und \square heisst nun \squaren
% replaced by \usepackage[free-standing-units=true,use-xspace=true]{siunitx} but at the beginning of the document

%Spannung
\DeclareMathOperator{\V}{\volt}
\DeclareMathOperator{\mV}{\milli \volt}
\DeclareMathOperator{\uV}{\micro \volt}

%Strom
\DeclareMathOperator{\A}{\ampere}
\DeclareMathOperator{\mA}{\milli \ampere}
\DeclareMathOperator{\uA}{\micro \ampere}
\DeclareMathOperator{\nA}{\nano \ampere}

%Zeit
\DeclareMathOperator{\s}{\second}
\DeclareMathOperator{\ms}{\milli \second}
\DeclareMathOperator{\us}{\micro \second}
\DeclareMathOperator{\ns}{\nano \second}

%Kapazitaet
\DeclareMathOperator{\mF}{\milli \farad}
\DeclareMathOperator{\uF}{\micro \farad}
\DeclareMathOperator{\nF}{\nano \farad}
\DeclareMathOperator{\pF}{\pico \farad}
\DeclareMathOperator{\fF}{\femto \farad}

%Induktivitaet
\DeclareMathOperator{\mH}{\milli \henry}
\DeclareMathOperator{\uH}{\micro \henry}
\DeclareMathOperator{\nH}{\nano \henry}

%Widerstand
\DeclareMathOperator{\MO}{\mega \ohm}
\DeclareMathOperator{\kO}{\kilo \ohm}
\DeclareMathOperator{\mO}{\milli \ohm}
\DeclareMathOperator{\Ohm}{\ohm}
%Strecke
\DeclareMathOperator{\km}{\kilo \meter}
\DeclareMathOperator{\cm}{\centi \meter}
\DeclareMathOperator{\mm}{\milli \meter}

%Frequenz
\DeclareMathOperator{\GHz}{\giga \hertz}
\DeclareMathOperator{\MHz}{\mega \hertz}
\DeclareMathOperator{\Hz}{\hertz}
\DeclareMathOperator{\kHz}{\kilo \hertz}
\DeclareMathOperator{\mHz}{\milli \hertz}

%Leistung
\DeclareMathOperator{\kW}{\kilo \watt}
\DeclareMathOperator{\mW}{\milli \watt}
\DeclareMathOperator{\uW}{\micro \watt}
\DeclareMathOperator{\W}{\watt}

%Kreisfrequenz
\DeclareMathOperator{\rpers}{\radianpersecond}

%DeziBel
\DeclareMathOperator{\dB}{\deci \bel}
\DeclareMathOperator{\dBm}{\deci \bel \milli}

%Bit
\DeclareMathOperator{\Bit}{\text{Bit}}
\DeclareMathOperator{\kBit}{\text{kBit}}
\DeclareMathOperator{\MBit}{\text{MBit}}
\DeclareMathOperator{\Byte}{\text{Byte}}
\DeclareMathOperator{\kByte}{\text{kByte}}
\DeclareMathOperator{\MByte}{\text{MByte}}
\DeclareMathOperator{\ppm}{\text{ppm}}
